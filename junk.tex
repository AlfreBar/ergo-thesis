\documentclass[12pt,a4paper]{book}
\usepackage[utf8]{inputenc}
%\usepackage[T1]{fontenc}
\usepackage{amsthm}
\usepackage{amsmath}
\usepackage{amsfonts}
\usepackage{amssymb}
\usepackage{graphicx}
\usepackage{physics}
\usepackage{cite}
\usepackage{caption}
\usepackage{subcaption}
%\usepackage{hyperref}
\usepackage{comment}
\usepackage{epsfig}
\usepackage{latexsym}
\usepackage{color} 
\usepackage{todonotes}
\theoremstyle{definition}
\newtheorem{definition}{Definition}[section]
\newtheorem{theorem}{Theorem}[section]
\newtheorem{lemma}[theorem]{Lemma}

\begin{document}
	JUNK \\
	

	For $\lambda \rightarrow 0 $ : 
	\begin{equation}
		\begin{array}{l}
			g_{l}=-\left(\frac{1+\gamma}{2} L_{l+1}+\frac{1-\gamma}{2} L_{l-1}\right) \quad l \in \text { odd } \\
			g_{l}=0 \quad l \in \text { even }
		\end{array}
	\end{equation}
	where:
	\begin{equation}
		L_{l}=\frac{2}{\pi} \int_{0}^{\pi / 2} d \phi \frac{\cos \phi l}{\sqrt{\cos ^{2} \phi+\gamma^{2} \sin ^{2} \phi}}
	\end{equation}
For $E(\rho)$:
\begin{equation}
	\begin{aligned}
		E(\rho)=&r_1 [e_1(r_1|e_1)+e_4(r_1|e_4)] +\\
		&r_2 e_3 +\\
		&r_3 e_2 + \\
		&r_4 [e_1(r_4|e_1)+e_4(r_4|e_4)] \\
	\end{aligned}
\end{equation}
\begin{equation}
	\begin{aligned}
		E(\rho)=&r_1 [e_1(r_1|e_1)+e_4(r_1|e_4)] +\\
		&1/8(1-\nu_-)(1+\nu_+)+\\
		&-1/8(1+\nu_-)(1-\nu_+)+ \\
		&r_4 [e_1(r_4|e_1)+e_4(r_4|e_4)] \\
	\end{aligned}
\end{equation}
We should express the eigenvalues of $H$ in terms of the nus maybe.
Or better we can do the trace multiplication in some smart way.\\

Without Identity:
\begin{equation}
	\Tr \left(\sigma_{a} \sigma_{b} \sigma_{c} \sigma_{d}\right)=2\left(\delta_{a b} \delta_{c d}-\delta_{a c} \delta_{b d}+\delta_{a d} \delta_{b c}\right)
\end{equation}
\begin{equation}
	\Tr\left(\sigma_{a} \sigma_{b} \sigma_{c}\right)=2 i \varepsilon_{a b c}
\end{equation}
\begin{equation}
	\Tr\left(\sigma_{a} \sigma_{b}\right)=2 \delta_{a b}
\end{equation}

$E(\rho)=\Tr[H \rho]$

\begin{equation}
	\rho_2=\frac{1}{4} [ ( \sigma_1^0 \sigma_2^0 ) -g_{-1} \left(\sigma_1^x \sigma_2^x\right) - g_{1} \left(\sigma_1^y \sigma_2^y\right) + (g_{0}^2-g_1g_{-1}) \left(\sigma_1^z \sigma_2^z\right)+g_0\left(\sigma_1^0 \sigma_2^z+\sigma_1^z \sigma_2^0\right) ] 
\end{equation}
\begin{equation}
	H=-\frac{1}{2} \left(\sigma^{x} \sigma^{x}+\lambda\sigma^{z} \sigma^0+\lambda\sigma^0\sigma^{z}\right) 
\end{equation}

4 terms: xxxx , yyxx, zzxx  2 for everyone \\

3 terms: 0zxx, z0xx, xxz0, xx0z, yyz0, yy0z, zzz0, zz0z  0 for everyone\\

2 terms 0zz0 0z0z z0z0 z00z 00xx  2 for everyone\\

1 term 00z0 000z =0 for everyone \\

Result, not zero:\\
xxxx , yyxx, zzxx, 0zz0 0z0z z0z0 z00z 00xx \\
$-1/4(-g_{-1}-g_1+ g_{0}^2-g_1g_{-1}+ 4g_0 \lambda + 1)$





\begin{equation}
	\left.\left(
	\begin{array}{cccc}
		\sqrt{4 \lambda ^2+1}+2 \lambda  & 0 & 0 & 1 \\
		0 & 1 & 1 & 0 \\
		0 & -1 & 1 & 0 \\
		2 \lambda -\sqrt{4 \lambda ^2+1} & 0 & 0 & 1 \\
	\end{array}
	\right)\right\}
\end{equation}


\begin{equation}
	\left(
	\begin{array}{cccc}
		\sqrt{4 g_0^2+(g_1-g_{-1})^2}+2 g_0 & 0 & 0 & g_1-g_{-1}  \\
		0 & -1 & 1 & 0 \\
		0 & 1 & 1 & 0 \\
		\sqrt{4 g_0^2+(g_1-g_{-1})^2}-2 g_0 & 0 & 0 & g_1-g_{-1} \\
	\end{array}
	\right)
\end{equation}
Calling $\alpha=\frac{g_{-1}-g_1}{2g_0}$:
\begin{equation}
	\left(
	\begin{array}{cccc}
		-\sqrt{1+\alpha^2}-1 & 0 & 0 & \alpha \\
		0 & -1 & 1 & 0 \\
		0 & 1 & 1 & 0 \\
		\sqrt{1+\alpha^2}-1 & 0 & 0 & \alpha \\
	\end{array}
	\right)
\end{equation}


Eigenstates in the same order:
\begin{equation}
	\left(
	\begin{array}{cccc}
		-\frac{\sqrt{4 g_0^2+(g_1-g_{-1})^2}+2 g_0}{g_{-1}-g_1} & 0 & 0 & 1 \\
		0 & -1 & 1 & 0 \\
		0 & 1 & 1 & 0 \\
		-\frac{2 g_0-\sqrt{4 g_0^2+(g_1-g_{-1})^2}}{g_{-1}-g_1} & 0 & 0 & 1 \\
	\end{array}
	\right)
\end{equation}


\end{document}